\section{Introduction}
In this part of the report, the fundamental concepts regarding matrix discplacement
methods will be briefly summerized, both for the sake of completeness and to 
introduce the notation that will be used throughout the text. The topics are 
explained in the order given below:
\begin{itemize}
   \item Fundamentals of elasticity
   \item Energetical approaches to solve problems of elasticity
   \item Stress based approach (Flexibility Method)
   \item Displacement based approach (Stiffness Method)
   \item Displacement method in matrix form
\end{itemize}   
%
% Fundamentals of Elasticity
%
\subsection{Fundamentals of Elasticity}
Elasticity is one great branch of mechanics that deal with deformable bodies which
undergo deformations as a result of external effects, and in which the deformations
vanish once the externally applied forces are removed. In addition to this
reversibility, if the relationship between stresses and strains are linear, than it 
is called 'Linear Elasticity', which will be one of the key assumptions of the 
fornulation that follows. The linear elastic constitutive law can be simply stated as:

\begin{equation}
	\sigma = E \epsilon
\end{equation}
for the uniaxial case, where the stress and strain are denoted by $\sigma$ and 
$\epsilon$  respectively, and E is the Young's Modulus of the material. 

For plane stress, the law should be restated in matrix form as follows:

\begin{equation*}
	\bm{\sigma = D \epsilon}
\end{equation*}
\begin{equation*}
	\begin{bmatrix}
		\sigma_{xx} \\
		\sigma_{yy} \\
		\sigma_{xy} \\
	\end{bmatrix}
	= \frac{E}{1-\nu^2}
	\begin{bmatrix}
		1 & \nu & 0 \\
		\nu & 1 & 0 \\
		0 & 0 & 1-\nu
	\end{bmatrix}
	\begin{bmatrix}
		\epsilon_{xx} \\
		\epsilon_{yy} \\
		\epsilon_{xy} \\
	\end{bmatrix}
\end{equation*}

Now let's try a matrix in which there exists some submatrices:
\begin{equation*}
	\begin{bmatrix}
		\bm{A_{11}^{'}} & \bm{A_{12}^{'}} \\
		\bm{0} & \bm{A_{22}^{'}} \\
	\end{bmatrix}
\end{equation*}

Adding images to the document: This is a sample moment diagram.
\begin{figure}
	\includegraphics{img/md1.jpg}
	\centering
	\caption{Sample moment diagram}
	\label{fig:md1}
\end{figure}
Figure \ref{fig:md1} is a sample moment diagram.

